\documentclass[12pt]{article}
\usepackage{pmmeta}
\pmcanonicalname{LinearProgramming}
\pmcreated{2013-03-22 13:41:41}
\pmmodified{2013-03-22 13:41:41}
\pmowner{mathcam}{2727}
\pmmodifier{mathcam}{2727}
\pmtitle{linear programming}
\pmrecord{12}{34369}
\pmprivacy{1}
\pmauthor{mathcam}{2727}
\pmtype{Topic}
\pmcomment{trigger rebuild}
\pmclassification{msc}{90C05}
\pmsynonym{LP}{LinearProgramming}
%\pmkeywords{linear programming}
\pmrelated{DualityInMathematics}
\pmdefines{linear programming problem}
\pmdefines{objective function}
\pmdefines{feasible region}
\pmdefines{feasible}

\endmetadata

% this is the default PlanetMath preamble.  as your knowledge
% of TeX increases, you will probably want to edit this, but
% it should be fine as is for beginners.

% almost certainly you want these
\usepackage{amssymb}
\usepackage{amsmath}
\usepackage{amsfonts}

% used for TeXing text within eps files
%\usepackage{psfrag}
% need this for including graphics (\includegraphics)
%\usepackage{graphicx}
% for neatly defining theorems and propositions
%\usepackage{amsthm}
% making logically defined graphics
%%%\usepackage{xypic}

% there are many more packages, add them here as you need them

% define commands here
\begin{document}
\PMlinkescapeword{polynomial}
\PMlinkescapeword{exponential}
\PMlinkescapeword{theory}
\PMlinkescapeword{unbounded}
\PMlinkescapeword{states}
\PMlinkescapeword{function}
A \emph{linear programming problem}, or \emph{LP}, 
is a problem of optimizing a given
linear objective function over some polyhedron.  The standard
maximization LP, sometimes called the
primal problem, is
\begin{align*}
\text{maximize\ }   & c^Tx\  \\
\text{s.t.\ }       & Ax\le b\tag{P}\label{eq:primal} \\
                    &  x\ge 0
\end{align*}
Here $c^Tx$ is the objective function and the remaining conditions
define the polyhedron which is the feasible region over which the
objective function is to be optimized.  The dual of
$({\ref{eq:primal}})$ is the LP
\begin{align*}
\text{minimize\ }  & y^Tb\ \\
\text{s.t.\ }      & y^TA\ge c^T\tag{D}\label{eq:dual} \\
                   &    y\ge 0
\end{align*}

The linear constraints for a linear programming problems define a convex polyhedron, called the \emph{feasible region} for the problem.  The weak duality theorem states that if $\hat{x}$ is feasible (i.e. lies in the feasible region) for
$(\ref{eq:primal})$ and $\hat{y}$ is feasible for $(\ref{eq:dual})$,
then $c^T\hat{x}\le\hat{y}^Tb$.  This follows readily from the above:
\[c^T\hat{x}\le(\hat{y}^TA)\hat{x}=\hat{y}^T(A\hat{x})\le y^Tb.\]
The strong duality theorem states that if both LPs are feasible,
then the two objective functions have the same optimal value.  As a
consequence, if
either LP has unbounded objective function value, the other must
be infeasible.  It is also possible for both LP to be infeasible.

The \PMlinkname{simplex method}{SimplexAlgorithm} of G. B. Dantzig is 
the algorithm
most commonly used to solve LPs; in practice it runs in polynomial time,
but the worst-case running time is exponential.  Two polynomial-time
algorithms for solving LPs are the ellipsoid method of L. G. Khachian
and the interior-point method of N. Karmarkar.

\begin{thebibliography}{3}
\bibitem{cite:C}
Chv\'{a}tal, V.,  \emph{Linear programming}, W. H. Freeman and Company, 1983.
\bibitem{cite:CLRS}
Cormen, T. H., Leiserson, C. E., Rivest, R. L., and C. Stein,
\emph{Introduction to algorithms}, MIT Press, 2001.
\bibitem{cite:KV}
Korte, B. and J. Vygen,  \emph{Combinatorial optimization: theory and
algorithms}, Springer-Verlag, 2002.
\end{thebibliography}
%%%%%
%%%%%
\end{document}
