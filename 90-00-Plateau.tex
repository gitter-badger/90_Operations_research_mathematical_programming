\documentclass[12pt]{article}
\usepackage{pmmeta}
\pmcanonicalname{Plateau}
\pmcreated{2013-03-22 12:59:48}
\pmmodified{2013-03-22 12:59:48}
\pmowner{bshanks}{153}
\pmmodifier{bshanks}{153}
\pmtitle{plateau}
\pmrecord{8}{33374}
\pmprivacy{1}
\pmauthor{bshanks}{153}
\pmtype{Definition}
\pmcomment{trigger rebuild}
\pmclassification{msc}{90-00}
\pmclassification{msc}{26B12}
\pmrelated{Extremum}

\endmetadata

% this is the default PlanetMath preamble.  as your knowledge
% of TeX increases, you will probably want to edit this, but
% it should be fine as is for beginners.

% almost certainly you want these
\usepackage{amssymb}
\usepackage{amsmath}
\usepackage{amsfonts}

% used for TeXing text within eps files
%\usepackage{psfrag}
% need this for including graphics (\includegraphics)
%\usepackage{graphicx}
% for neatly defining theorems and propositions
%\usepackage{amsthm}
% making logically defined graphics
%%%\usepackage{xypic}

% there are many more packages, add them here as you need them

% define commands here
\begin{document}
A \emph{plateau} of a function is a region where a function has constant value.

More formally, let $U$ and $V$ be topological spaces. A plateau for a scalar function $f: U \to V$ is a path-connected set of points $P \subseteq U$ such that for some $y$ we have

\begin{equation}
\forall p \in P, f(p) = y
\end{equation}

Please take note that this entry is not authoritative. If you know of a more standard definition of ``plateau'', please contribute it, thank you.
%%%%%
%%%%%
\end{document}
