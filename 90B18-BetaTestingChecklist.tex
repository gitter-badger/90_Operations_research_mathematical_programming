\documentclass[12pt]{article}
\usepackage{pmmeta}
\pmcanonicalname{BetaTestingChecklist}
\pmcreated{2013-05-16 10:55:58}
\pmmodified{2013-05-16 10:55:58}
\pmowner{unlord}{1}
\pmmodifier{unlord}{1}
\pmtitle{Beta testing checklist}
\pmrecord{2}{87386}
\pmprivacy{1}
\pmauthor{unlord}{1}
\pmtype{Topic}
\pmclassification{msc}{90B18}

\endmetadata

% this is the default PlanetMath preamble.  as your knowledge
% of TeX increases, you will probably want to edit this, but
% it should be fine as is for beginners.

% almost certainly you want these
\usepackage{amssymb}
\usepackage{amsmath}
\usepackage{amsfonts}

% need this for including graphics (\includegraphics)
\usepackage{graphicx}
% for neatly defining theorems and propositions
\usepackage{amsthm}

% making logically defined graphics
%\usepackage{xypic}
% used for TeXing text within eps files
%\usepackage{psfrag}

% there are many more packages, add them here as you need them

% define commands here

\begin{document}
Instructions for Beta Testers:

Over the next 10 days or so, please try all of these features:

\begin{description}
\item[Articles] You can make new ones, adopt items in the orphanage, share your articles with other people via editing groups \ldots
\item[Comments] Add comments on existing articles, post in the forums \ldots
\item[Corrections] If you see mistakes in other people's articles, you can point these out via a correction; if someone finds a mistake in your articles, they can add a correction which you should address in a timely fashion if possible!
\item[Questions] Ask questions if you get stuck, answer other people's questions if you can.  You can also use the ``Questions'' feature to make requests for articles that don't exist yet \ldots
\item[Problems, Solutions, Reviews] Try uploading some new problems, solving some existing problems, giving reviews to other people \ldots
\item[Groups] I've created a group for us and I'm adding this article to the group, so you should all be able to edit it.  You can create new groups (e.g. Mathematics for Robot Designers, Approximation theory, Computer Science exercises, General Relativity Study Group, Hyperbolic Geometry Editors, etc.) and add content and people to them.  To add content to a group, you need to have permission to edit the content in question.
\item[Collections] This is a new feature for creating reading lists and problem sets.  You \emph{don't} need editing permission to add something to a collection.  
\end{description}

Regarding LaTeX:  See the existing \href{http://planetmath.org/TeXMathematicalFormulaQuickReference}{site doc} for some useful quickstart tips, there are a few pointers in the \href{http://planetmath.org/theplanetmathfaq1}{FAQ}.  WikiBooks also has a \href{http://en.wikibooks.org/wiki/LaTeX}{nice guide} going into more detail.  Oh, and you can \emph{view the source} of PlanetMath articles to see how the author wrote it!

If you get stuck with any of these things, leave a comment on this article or ping me by email or Skype, and I'll try to help.  If you notice other problems or little pink boxes showing up with error messages, please email me with a link or screenshot and I'll try to fix it ASAP.  Bugs can also be reported in the \href{http://planetmath.org/forums/planetary-bugs}{Planetary Bugs Forum}.

Talk to you again soon! ~Joe
\end{document}
